%% example.tex
%% Copyright 2023- J.P. van Leenen
%%
%% LPPL 1.3c+

\documentclass[aspectratio=169]{beamer}
% Options:
% - noslidenumbering (no options), disables slide numbering
% - style=[archaeology, fgga, humanities, law, medicine, science, social], changes accent colour to match faculty
\usetheme[
    % % e.g.
    % noslidenumbering,
    % style=science,
]{lu}
% Set text in the left-corner of the footer, comment to disable.
\setlufootleft{Van Leenen --- Leiden is Great}

\usepackage{appendixnumberbeamer}
\graphicspath{{graphics/}}

\title{What Makes Leiden University\\Great}
\author{A Template by Jort van Leenen}
\institute{Based on the Official PowerPoint}

\begin{document}
    \begin{frame}[plain]
        \titlepage
    \end{frame}

    \begin{frame}{Some Block and Styling Examples}
        \begin{definition}
            \textbf{Leiden University} is the \emph{oldest} Dutch institution of higher education.
        \end{definition}

        \begin{alertblock}{Housing}
            Start searching for a place to stay \alert{as soon as possible}!
        \end{alertblock}

        \begin{exampleblock}{Open}
            Everyone is welcome!
        \end{exampleblock}
    \end{frame}

    \begin{frame}[fragile]{\texttt{\textbackslash accent} Usage}
        \texttt{\textbackslash accent} behaves exactly like \texttt{\textbackslash alert}. However, one of the theme's colours can be specified as an option to allow for easy highlighting.

        \begin{table}[]
            \centering
            \begin{tabular}{c|c}
                \hline
                \accent{highlighted}             & \verb|\accent{highlighted}|             \\
                \accent[lightblue]{highlighted}  & \verb|\accent[lightblue]{highlighted}|  \\
                \accent[midblue]{highlighted}    & \verb|\accent[midblue]{highlighted}|    \\
                \accent[darkblue]{highlighted}   & \verb|\accent[darkblue]{highlighted}|    \\
                \accent[red]{highlighted}        & \verb|\accent[red]{highlighted}|        \\
                \accent[lightgreen]{highlighted} & \verb|\accent[lightgreen]{highlighted}| \\
                \accent[darkgreen]{highlighted}  & \verb|\accent[darkgreen]{highlighted}|  \\
                \accent[turquoise]{highlighted}  & \verb|accent[turquoise]{highlighted}|   \\
                \accent[violet]{highlighted}     & \verb|\accent[violet]{highlighted}|     \\
                \hline
            \end{tabular}
        \end{table}
    \end{frame}

    \begin{frame}{List Examples}{And a Subtitle Example as Bonus}
        \begin{columns}[onlytextwidth]%
        \begin{column}{.5\textwidth}%
        \textbf{Itemize}
        
            \begin{itemize}
                \item This is the first item.
                \item This is the second item.
                \begin{itemize}
                    \item This is the first subitem.
                    \begin{itemize}
                        \item This is the first subsubitem.
                    \end{itemize}
                \end{itemize}
            \end{itemize}
        \end{column}%
        \begin{column}{.5\textwidth}%
            \textbf{Enumerate}

            \begin{enumerate}
                \item This is the first item.
                \item This is the second item.
                \begin{enumerate}
                    \item This is the first subitem.
                    \begin{enumerate}
                        \item This is the first subsubitem.
                    \end{enumerate}
                \end{enumerate}
            \end{enumerate}
        \end{column}%
    \end{columns}%
    \end{frame}

    \begin{frame}[fragile]{Listings Template}
        Various styles have been provided with this theme for external tools. See the \texttt{styles} folder for external library and application styles. LaTeX-related styles have been added to \texttt{beamerthemelu.sty}. As an example, take the following code listing to which the Listings style \texttt{lu} has been applied.

        \begin{lstlisting}[language=python, style=lu]
print('Hello world!') # An example of Hello World in Python
        \end{lstlisting}
    \end{frame}

    \begin{frame}[c, standout]{Standout}
        This template reverses the colour usage. It could, for example, be used to review or recap information or draw special attention to a particular aspect.
    \end{frame}

    \begin{frame}[plain]
        \Closure
            {Thank you}{Questions?}
    \end{frame}

    % \appendix
    % \begin{frame}[allowframebreaks]{References}
    %     \printbibliography
    % \end{frame}
\end{document}
